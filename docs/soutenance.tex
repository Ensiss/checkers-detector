\documentclass{beamer}
\usetheme{Berkeley}\usecolortheme{whale}
%\usetheme{Hannover}\usecolortheme{beaver}
\usepackage[utf8]{inputenc}
\usepackage[french]{babel}
\usepackage[T1]{fontenc}
\usepackage{array}
\title[Algorithmes Animés]{L3I1 - Algorithmes Animés}
\author[Ecormier Lacambre]{Pierre Ecormier, Ghjuvan Lacambre}
\institute[Paris Descartes]{Université René Descartes, Paris V\\UFR de Mathématiques et Informatique}
\date{\today}
\begin{document}
\sffamily
    \maketitle
    \begin{frame}
        \frametitle{Table of Contents}
        \tableofcontents
    \end{frame}
    \section{Introduction}
    \begin{frame}
      \frametitle{Introduction}
      \begin{itemize}
      \item Détecter un damier dans une image
      \item Gestion de la rotation, perspective, etc
      \item Gestion des damiers de 8 ou 10 cases
      \end{itemize}
    \end{frame}
    \section{Extraction du damier}
    \subsection{Détection}
    \begin{frame}
      \frametitle{Détection}
      \begin{itemize}
      \item Canny
      \item Hough Transform
      \item Algorithme de pattern matching maison
      \end{itemize}
    \end{frame}
    \subsection{Normalisation}
    \begin{frame}
      \frametitle{Normalisation}
      \begin{itemize}
      \item Bilinear interpolation
      \item Redressement de perspective
      \end{itemize}
    \end{frame}
    \section{Détection des pions}
    \subsection{Détection du fond}
    \begin{frame}
      \frametitle{Détection du fond}
      \begin{itemize}
      \item Détection de l'ordre du damier
      \item Blanc-noir ou noir-blanc
      \item Facilite le traitement suivant
      \end{itemize}
    \end{frame}
    \subsection{Détection des pions}
    \begin{frame}
      \frametitle{Détection des pions}
      \begin{itemize}
      \item k-means
      \item CAH avec lien minimal
      \item Différences d'histogrammes
      \end{itemize}
    \end{frame}
    \section{Conclusion}
    \subsection{Défauts}
    \begin{frame}
      \frametitle{Défauts}
      \begin{itemize}
      \item Images comprenant des textures (herbe, moquette, etc)
      \item Pions noirs sur noir ou blanc sur blanc
      \end{itemize}
    \end{frame}
    \subsection{Améliorations}
    \begin{frame}
      \frametitle{Améliorations}
      \begin{itemize}
      \item Détection des pions avec Hough Circles (peut apporter des échecs sur certains cas)
      \item Plus de pré-traitement (suppression de textures, etc)
      \end{itemize}
    \end{frame}
  \end{document}
